\documentclass[UTF-8,twoside,cs4size]{ctexart}
\usepackage[dvipsnames]{xcolor}
\usepackage{amsmath}
\usepackage{amssymb}
\usepackage{geometry}
\usepackage{setspace}
\usepackage{xeCJK}
\usepackage{ulem}
\usepackage{pstricks}
\usepackage{pstricks-add}
\usepackage{bm}
\usepackage{mathtools}
\usepackage{breqn}
\usepackage{mathrsfs}
\usepackage{esint}
\usepackage{textcomp}
\usepackage{upgreek}
\usepackage{pifont}
\usepackage{tikz}
\usepackage{circuitikz}
\usepackage{caption}
\usepackage{tabularx}
\usepackage{array}
\usepackage{pgfplots}
\usepackage{multirow}
\usepackage{pgfplotstable}
\usepackage{mhchem}

\newcolumntype{Y}{>{\centering\arraybackslash}X}
\geometry{a4paper,centering,top=0.75cm,bottom=2.54cm,left=2cm,right=2cm}
\pagestyle{plain}
\captionsetup{font=small}

%\CTEXsetup[name={,.}]{section}
\CTEXsetup[format={\raggedright\bfseries\noindent\zihao{3}}]{section}
\CTEXsetup[format={\raggedright\bfseries\quad\large}]{subsection}
\CTEXsetup[format={\raggedright\bfseries\qquad}]{subsubsection}
\renewcommand\thefootnote{\ding{\numexpr171+\value{footnote}}}

\setstretch{1.5}

\setCJKfamilyfont{boldsong}[AutoFakeBold = {2.17}]{SimSun}
\newcommand*{\boldsong}{\CJKfamily{boldsong}}
%\DeclareMathOperator\dif{d\!}
\newcommand*{\me}{\mathop{}\!\mathrm{e}}
\newcommand*{\mpar}{\mathop{}\!\partial}
\newcommand*{\dif}{\mathop{}\!\mathrm{d}}
\newcommand*{\tab}{\indent}
\newcommand*{\mcelsius}{\mathop{}\!{^\circ}\mathrm{C}}
\renewcommand*{\Im}{\mathrm{Im}\,}

\begin{document}
	\begin{flushright}
		\zihao{2}{分组号:3-07}
	\end{flushright}
	
	\noindent{\zihao{-2}\boldsong\bfseries 《\,\, 基\,\, 础\,\, 物\,\, 理\,\, 实\,\, 验\,\, 》\,\, 实\,\, 验\,\, 报\,\, 告\,\, }
	
	\noindent{\kaishu 实验名称\uline{\quad\qquad\qquad\,\,\,$ RLC $电路的谐振与暂态过程\qquad\qquad\qquad}指导教师\uline{\qquad\,\,\,袁跃峰\,\,\,\qquad}}
	
	\noindent\textit{姓\qquad 名\uline{\,\,\, 桂庭辉\,\,\,}\,学号\uline{\,\,\,{\upshape2019K8009929019}\,\,\,}\,专\qquad 业\uline{\,\,\,计算机科学与技术\,\,\,}\,班级\uline{\,\,\,\upshape{03}\,\,\,}\,座号\uline{\,\,\,\upshape{6}\,\,\,}}
	
	\noindent\textit{实验日期\uline{\,\,{\upshape 2020}\,\,}年\uline{\,\,{\upshape 12}\,\,}月\uline{\,\,{\upshape30}\,\,}日\,\,实验地点\uline{\,\,\,教学楼{\upshape709}\,\,\,}调课/补课\uline{\,\,\,$ \square $是\,\,\,}成绩评定\uline{\,\,\,\quad\qquad\qquad}}
	
	\begin{table}[h]
		\centering
		\psset{linewidth=2pt}
		\begin{pspicture}(-1,-0.1)(1,0.1)
		\psline(-9,0)(9,0)
		\end{pspicture}
	\end{table}

	\section{实验目的}
	1.通过观察示波器中的波形研究$ RLC $电路的谐振现象。
	
	2.了解$ RLC $电路的相频特性与振幅特性。
	
	3.通过示波器中的波形观察$ RLC $串联电路的暂态过程。
	
	\section{实验仪器与用具}
	标准电感、标准电容,$ 100\,\Omega $标准电阻,电阻箱,电感箱,函数发生器,示波器,数字多用表,导线等。
	
	\section{实验原理}
	\subsection{$ RLC $串联谐振电路}
	\begin{figure}[!h]
		\centering
		\begin{circuitikz}
			\draw (6,0)
			to[sV,a=$ \tilde u_s $] (0,0)
			to[short] (0,-3)
			to[european resistor,l=$ R $] (2,-3)
			to[american inductor,l=$ L $] (4,-3)
			to[C,l=$ C $] (6,-3)
			to[short] (6,0);
			\draw[->] (5,0)--(5.1,0)node[above]{$ \tilde i $};
			\draw[<-] (0.1,-1.5)--(2.7,-1.5);
			\draw[->] (3.3,-1.5)--(5.9,-1.5);
			\node at(3,-1.5) {$ \tilde u $};
			\draw (6,-3.5)--(6,-4.5);
			\draw (4,-3.5)--(4,-4.5);
			\draw (2,-3.5)--(2,-4.5);
			\draw (0,-3.5)--(0,-4.5);
			\draw[<-] (0.05,-4)--(0.7,-4);
			\draw[<-] (2.05,-4)--(2.7,-4);
			\draw[<-] (4.05,-4)--(4.7,-4);
			\draw[->] (1.3,-4)--(1.95,-4);
			\draw[->] (3.3,-4)--(3.95,-4);
			\draw[->] (5.3,-4)--(5.95,-4);
			\node at(1,-4) {$ \tilde u_R $};
			\node at(3,-4) {$ \tilde u_L $};
			\node at(5,-4) {$ \tilde u_C $};
		\end{circuitikz}
		\caption{$ RLC $串联谐振电路}
	\end{figure}
	
	$ RLC $串联谐振电路如上图所示,利用复数法\footnote{本实验报告中复数法记法约定$ \tilde x $表示复变量,$ x $表示其模长,虚单位为$ j=\sqrt{-1} $。}分析电路可求得电容$ C $、电感$ L $、电阻$ R $的复阻抗分别为
	\[\tilde Z_C=\frac{1}{j\omega C},\quad \tilde Z_L=j\omega L,\quad\tilde Z_R=R\]
	进而可求得电路的总复阻抗与总阻抗:
	\[\tilde Z=R+j\left(\omega L-\frac{1}{\omega C}\right)\,\Longrightarrow\,Z=\left|\tilde Z\right|=\sqrt{R^2+\left(\omega L-\frac{1}{\omega C}\right)^2}\]
	电流$ i $的大小为
	\[i=\frac uZ=\frac{1}{\sqrt{R^2+\left(\omega L-\frac{1}{\omega C}\right)^2}}\]
	路端电压$ u $与电流$ i $的相位差为
	\[\varphi=\arctan\frac{\omega L-\frac{1}{\omega C}}{R}\]
	以上诸式中$ \omega=2\pi f $为交变电压$ u $的角频率,$ f $为其频率。可见$ Z,\,\varphi,i $均为$ f $的函数,即当电路中其他元件参量确定的情况下,电路特性将完全取决于频率的大小。
	
	\begin{figure}[!h]
		\centering
		\includegraphics*[]{figure4.pdf}
		\caption{$ RLC $串联电路的频率特性}
	\end{figure}
	
	1.当$ \Im\tilde Z=0 $,即$ \omega L-\frac{1}{\omega C}=0 $时,总阻抗呈电阻性,且总阻抗达到最小值$ Z_0=R $,电压与电路的相位差$ \varphi=0 $,电流达到最大值$ i_\max=\frac uR $,这种状态称为串联谐振。此时的角频率$ \omega_0 $称为谐振角频率,相应地此时频率$ f_0 $称为谐振频率,其具体大小为
	\[\omega_0=\frac{1}{\sqrt{LC}},\quad f_0=\frac{1}{2\pi\sqrt{LC}}\]
	谐振时,有
	\[u_L=i_\max Z_L=\frac{\omega_0L}{R}u,\quad \frac{u_L}{u}=\frac{\omega_0L}{R}=\frac1R\sqrt{\frac LC}\]
	而
	\[u_C=i_\max Z_C=\frac{u}{R\omega_0C},\quad\frac{u_C}{u}=\frac{1}{R\omega_0C}=\frac1R\sqrt{\frac LC}\]
	令
	\[Q=\frac{u_L}{u}=\frac{u_C}{u}=\frac{\omega_0L}{R}=\frac{1}{R\omega_0C}\]
	$ Q $称为谐振电路的品质因数,是标志和衡量谐振电路性能优劣的重要参量。$ Q $可衡量的电路性能有:
	
	(1)储耗能特性:$ Q $值越大,相对的储能耗能越小,储能效率越高;
	
	(2)电压分配特性:谐振时$ u_L=u_C=Qu $,电感、电容上的电压均为总电压的$ Q $倍,因此有时称串联谐振为电压谐振。利用电压谐振,可在某些传感器、信息接收中显著提高灵敏度或效率,但在某些应用场合,它对系统和人员却具有一定不安全性,故而在设计与操作中应予以注意。
	
	(3)频率选择性:设$ f_1,\;f_2 $为谢振峰两侧$ i=\frac{i_\max}{\sqrt 2} $处对应频率,则$ \Delta f=f_2-f_1 $称为通频带宽度,简称带宽。不难证明有
	\[Q=\frac{f_0}{\Delta f}\]
	当$ Q $值越大时,带宽越窄,峰越尖锐,频率选择性越好。
	
	2.当$ \Im\tilde Z>0 $,即$ \omega L-\frac{1}{\omega C}>0 $时,总阻抗呈电感性,此时电压与电流相位差$ \varphi>0 $,交变电压频率$ f>f_0 $,并且随着$ f $的增大,$ \varphi $趋近于$ \frac\pi2 $,阻抗越大,电流越小。
	
	3.当$ \Im\tilde Z<0 $,即$ \omega L-\frac{1}{\omega C}<0 $时,总阻抗呈电感性,此时电压与电流相位差$ \varphi<0 $,交变电压频率$ f<f_0 $,并且随着$ f $的减小,$ \varphi $趋近于$ -\frac\pi2 $,阻抗越大,电流越小。
	
	\subsection{$ RLC $并联谐振电路}
	\begin{figure}[!h]
		\centering
		\begin{circuitikz}
			\draw (6,0)
			to[sV,a=$ \tilde u_s $] (0,0)
			to[short] (0,-1.5)
			to[european resistor,l=$ R' $,-*] (2,-1.5)
			to[european resistor,l=$ R $] (4,-1.5)
			to[american inductor,l=$ L $,-*] (6,-1.5)
			to[short,-.] (6,0);
			\draw (2,-1.5)
			to[short] (2,-3)
			to[C,l=$ C $] (6,-3)
			to[short] (6,-1.5);
			
			\draw[->] (5,0)--(5.1,0)node[above]{$ \tilde i $};
			\draw[<-] (5.7,-1.5)node[above]{$ \tilde i_L $}--(5.8,-1.5);
			\draw[<-] (5.5,-3)node[above]{$ \tilde i_C $}--(5.6,-3);
			
			\draw (0,-3.5)--(0,-4.5);
			\draw (2,-3.5)--(2,-4.5);
			\draw (6,-3.5)--(6,-4.5);
			\draw[<-] (0.05,-4)--(0.7,-4);
			\draw[->] (1.3,-4)--(1.95,-4);
			\node at(1,-4) {$ \tilde u_{R'} $};
			\draw[<-] (2.05,-4)--(3.7,-4);
			\draw[->] (4.3,-4)--(5.95,-4);
			\node at(4,-4) {$ \tilde u $};
		\end{circuitikz}
		\caption{$ RLC $并联谐振电路}
	\end{figure}
	
	利用复数法分析上图电路,电容$ C $、电感$ L $、电阻$ R $的复阻抗分别为
	\[\tilde Z_C=\frac{1}{j\omega C},\quad\tilde Z_L=j\omega L,\quad \tilde Z_R=R\]
	那么电路并联部分的总复阻抗为
	\[\tilde Z_p=\frac{1}{j\omega C+\frac{1}{R+j\omega L}}=\frac{R+j\omega L}{(1-\omega^2LC)+j\omega CR}\]
	从而可求得电路并联部分总电阻为
	\[Z_p=\left|\tilde Z_p\right|=\sqrt{\frac{R^2+(\omega R)^2}{(1-\omega^2LC)^2+(\omega CR)^2}}\]
	并联部分电压$ u $与干路电流$ i $的相位差为
	\[\varphi=\arctan\frac{\omega L-\omega C[R^2+(\omega L)^2]}{R}\]
	并联部分电压$ u $大小为
	\[u=iZ_p=\frac{u_{R'}}{R'}Z_p\]
	
	\begin{figure}[!h]
		\centering
		\includegraphics*[]{figure2.pdf}
		\caption{$ RLC $并联电路的频率特性}
	\end{figure}

	1.当$ \Im\tilde Z_p=0 $,即$ \varphi=0 $时,总阻抗呈纯电阻性,可求得其并联谐振的角频率$ \omega_p $与频率$ f_p $为
	\[\omega_p=\sqrt{\frac{1}{LC}-\left(\frac RL\right)^2}=\omega_0\sqrt{1-\frac{1}{Q^2}},\quad f_p=\frac{1}{2\pi}\sqrt{\frac{1}{LC}-\left(\frac RL\right)^2}\]
	即并联谐振频率$ f_p $与串联谐振频率$ f_0 $稍有不同,当$ Q\gg 1 $时,$ \omega_p\approx\omega_0 $,$ f_p\approx f_0 $。
	
	2.当$ f<f_p $时,$ \varphi>0 $,电流相位落后于电压,整个电路呈电感性。
	
	3.当$ f>f_p $时,$ \varphi<0 $,电流相位超前于电压,整个电路呈电容性。
	
	在谐振频率两侧区域,并联电路的电抗特性与串联电路相反。在$ f=f_p' $\footnote{$ f_p' $与$ f_p $稍有不同。}处总阻抗达到极大值,总电流达到极小值。而在$ f_p' $两侧,随$ f $偏离$ f_p' $越远,阻抗越小,电流越大。
	
	与串联谐振类似,可用品质因数
	\[Q_1=\frac{\omega_0L}{R}=\frac{1}{R\omega_0C},\quad Q_2=\frac{i_C}{i}\approx\frac{i_L}{i},\quad Q_3=\frac{f_0}{\Delta f}\]
	来标志并联谐振电路的性能优劣,有时也称并联谐振为电流谐振。
	
	\newpage
	\subsection{$ RLC $电路的暂态过程}
	\begin{figure}[!h]
		\centering
		\begin{circuitikz}
			\draw (0,2)
			to[battery1,a=$ E $] (0,0)
			to[short,-*] (1.5,0)
			to[short] (2,0)
			to[american inductor,a=$ L $] (5,0)
			to[C,l=$ C $] (5,2)
			to[european resistor,a=$ R $,-.] (2,2);
			\draw[thick] (2,2)node[above left]{$ S $}--(1.3,1.6);
			\draw (1.5,0)
			to[short] (1.5,1.6)node[right]{2};
			\draw (0,2)
			to[short,.-] (1.3,2)node[above]{1};
			\draw (5.5,0)--(6.5,0);
			\draw (5.5,2)--(6.5,2);
			\draw[<-] (6,0.05)--(6,0.7);
			\draw[->] (6,1.3)--(6,1.95);
			\node at(6,1) {$ u_C $};
		\end{circuitikz}
		\caption{$ RLC $串联振荡电路}
	\end{figure}
	
	在$ RLC $串联振荡电路中,开关拨向不同的端口,电路呈现为两种状态:当开关$ S $拨向1时,电源$ E $接入电路,为电容$ C $进行充电;当开关$ S $拨向2时,电容在$ RLC $串联电路中放电。在放电过程中,根据实际电路可列出常微分方程
	\[L\frac{\dif i}{\dif t}+Ri+u_C=0\]
	电容存储的电荷量$ q=Cu_C $,那么电路中的电流为
	\[i=\frac{\dif q}{\dif t}=C\frac{\dif u_C}{\dif t}\]
	将其代入电路常微分方程即得到关于$ u_C $的二阶齐次常微分方程
	\[LC\frac{\dif^2u_C}{\dif t^2}+RC\frac{\dif u_C}{\dif t}+u_C=0\]
	考虑初始条件即可得到方程组:
	\[\begin{cases*}
		LC\dfrac{\dif^2u_C}{\dif t^2}+RC\dfrac{\dif u_C}{\dif t}+u_C=0\\
		u_C=E\qquad(t=0)\\
		C\dfrac{\dif u_C}{\dif t}=0\qquad(t=0)
	\end{cases*}\]
	引入阻尼系数$ \zeta=\frac R2\sqrt{\frac CL} $,则可将方程组的解分为三种情况:
	
	1.当$ \zeta<1 $,即$ R^2<\frac{4L}{C} $时,阻尼不足,上述方程组的解为
	\[u_C=\sqrt{\frac{4L}{4L-R^2C}}E\me^{-t/\tau}\cos(\omega t+\varphi)\]
	其中时间常量$ \tau=\frac{2L}{R} $,衰减振动的角频率为$ \omega=\frac{1}{\sqrt{LC}}\sqrt{1-\frac{R^2C}{4L}} $。$ \tau $的大小决定了振幅衰减的快慢,$ \tau $越小,振幅衰减越迅速。
	
	若$ R^2\ll\frac{4L}{C} $,振幅的衰减很缓慢,此时
	\[\omega\approx\frac{1}{\sqrt{LC}}=\omega_0\]
	近似为$ LC $电路自由振动,$ \omega_0 $为$ R=0 $时$ LC $回路的固有频率。衰减振动的周期为
	\[T=\frac{2\pi}{\omega}\approx2\pi\sqrt{LC}\]
	
	
	\begin{figure}[!h]
		\centering
		\includegraphics*[]{figure6.pdf}
		\caption{$ RLC $暂态过程中的三种阻尼曲线}
	\end{figure}
	2.当$ \zeta>1 $,即$ R^2>\frac{4L}{C} $时,对应过阻尼状态,方程组的解为
	\[u_C=\sqrt{\frac{4L}{R^2C-4L}}E\me^{-\alpha t}\sinh(\beta t+\varphi)\]
	其中$ \alpha=\frac{R}{2L},\;\beta=\frac{1}{\sqrt{LC}}\sqrt{\frac{R^2C}{4L}-1} $。此时振幅将缓慢地衰减为0。若固定$ L,\;C $。
	
	3.当$ \zeta=1 $,即$ R^2=\frac{4L}{C} $时,对应临界阻尼状态,方程组的解为
	\[u_C=E\left(1+\frac t\tau\right)\me^{-t/\tau}\]
	其中$ \tau=\frac{2L}{R} $,其为从过阻尼到阻尼振动过渡的分界点。
	
	对于充电过程,考虑初始条件,电路方程组变为
	\[\begin{cases*}
		LC\dfrac{\dif^2u_C}{\dif t^2}+RC\dfrac{\dif u_C}{\dif t}+u_C=E\\
		u_C=0\qquad(t=0)\\
		\dfrac{\dif u_C}{\dif t}=0\qquad(t=0)
	\end{cases*}\]

	当$ R^2<\frac{4L}{C} $时,方程组的解为
	\[u_C=E\left[1-\sqrt{\frac{4L}{4L-R^C}}\me^{-t/\tau}\cos(\omega t+\varphi)\right]\]
	
	当$ R^2>\frac{4L}{C} $时,方程组的解为
	\[u_C=E\left[1-\sqrt{\frac{4L}{R^2C-4L}}\me^{-\alpha t}\sinh(\beta t+\varphi)\right]\]
	
	当$ R^2=\frac{4L}{C} $时,方程组的解为
	\[u_C=E\left[1-\left(1+\frac t\tau\right)\me^{-t/\tau}\right]\]
	
	可以看出,充电过程与放电过程十分类似,只是最后趋向的平衡位置不同。
	
	\section{实验内容}
	\subsection{测$ RLC $串联电路的相频特性曲线和幅频特性曲线}
	取$ u_{pp}=2.0\,\mathrm V,\;L=0.1\,\mathrm H,\;C=0.05\,\mu\mathrm F,\;R=100\,\Omega $时,用示波器CH1、CH2通道分别观测$ RLC $串联电路的总电压$ u $和电阻两端电压$ u_R $。注意限制总电压峰值不超过$ 3.0\,\mathrm V $(或有效值不超过0.1\,V),防止串联谐振时产生有危险的高电压。
	
	1.调谐振,改变函数发生器的输出频率,通过CH1与CH2相位差为0,CH2幅度最大来判断谐振与否,记录谐振时的频率$ f_0 $.
	
	2.用万用表记录谐振时的电感、电容两端的电压$ u_L,\;u_C $,和电源路端电压$ u $并计算$ Q $值。
	
	3.保持CH1的幅度为$ 2\,\mathrm V $不变,按照建议的频率点测量CH1与CH2的相位差、CH2的幅度值,并绘制相频曲线和幅频曲线,即$ \varphi-f $图象、$ i-f $图象。
	
	\subsection{测$ RLC $并联电路的相频特性和幅频特性曲线}
	取$ u+u_{R'}=2.0\,\mathrm V,\;L=0.1\,\mathrm H,\;C=0.05\,\mu\mathrm F,\;R'=5\,\mathrm k\Omega $。为观测电感与电容并联部分的电压和相位,用CH1测量总电压,用CH2测量$ R' $两端电压,两通道测量电压值相减即为并联部分的电压$ u $,可通过示波器面板上的“MATH”键实现两通道波形相减。
	
	1.调节函数发生器频率,通过观察CH1$ - $CH2与CH2相位差为0,CH2的幅度最小来判断谐振点,记录此时的频率。
	
	2.保持CH1总电压幅度值2\,V不变(不同频率点需要调节函数发生器),按照建议的频率点测量CH1$ - $CH2与CH2的相位差,与CH1$ - $CH2、CH2的幅度值,绘制相频曲线与幅频曲线,即$ \varphi-f $图象、$ i-f $图象、$ u-f $图象。
	
	\subsection{观测$ RLC $串联电路的暂态过程}
	由函数发生器产生方波,为便于观察,需将方波的低电平调整至与示波器的扫描基线一致。由低电平到高电平相当于充电,由高电平到低电平相当于放电。函数发生器各参数可设置为:频率50\,Hz,电压峰峰值$ u_{pp}=2.0\,\mathrm V $,偏移1\,V。示波器CH1通道用于测量总电压,CH2用来测量电容两端电压$ u_C $,注意两个通道必须共地。实验中$ L=0.1\,\mathrm H,\;C=0.2\,\mu\mathrm F $.
	
	1.当$ R=0\,\Omega $时,测量$ u_C $波形;
	
	2.调节$ R $测得临界电阻$ R_C $,并与理论值比较;
	
	3.记录$ R=2\,\mathrm{k}\Omega,\;20\,\mathrm k\Omega $的$ u_C $波形。函数发生器频率可设置为250\,Hz($ R=2\,\mathrm k\Omega $)和20\,Hz($ R=20\,\mathrm k\Omega $).
	
	\section{实验结果与数据处理}
	\subsection{测$ RLC $串联电路的相频特性曲线和幅频特性曲线}
	根据实验电路的参数可计算得到$ Q $的理论标准值
	\[Q=\frac1R\sqrt\frac LC=14.142\]
	
	(1)调节函数发生器输出频率,当$ f=2.259\,\mathrm{kHz} $时,路端电压与电阻$ R $两端电压相位差趋近为0,即达到谐振状态,此时用数字多用表测得
	\[u=0.444\,\mathrm V,\quad u_C=6.84\,\mathrm{V},\quad u_L=6.90\,\mathrm V\]
	故而可计算得到电路的品质因数为
	\[Q_1=\frac{u_C}{u}=15.405,\quad Q_2=\frac{u_L}{u}=15.541\]
	
	(2)在实验讲义给出的参考频率下,保证路端电压$ u_{pp}=2.0\,\mathrm V $不变的情况下测得电压、电流相位差,以及相应的$ u_R $值记录如下页。
	\begin{table}[!h]
		\centering
		\renewcommand\arraystretch{1.5}
		\caption{$ RLC $串联谐振电路实验数据记录}
		\begin{tabularx}{\textwidth}{|c|Y|Y|Y|Y|Y|}
			\hline
			\textbf{函数发生器}$ f\;(\mathrm{kHz}) $&\textbf{1.88}&\textbf{2.00}&\textbf{2.08}&\textbf{2.15}&\textbf{2.19}\\
			\hline
			\textbf{电压电流相位差}$ \varphi $&$-0.451\pi$&$ -0.424\pi $&$ -0.366\pi $&$ -0.266\pi $&$ -0.193\pi $\\
			\hline
			\textbf{电阻电压}$ U_R\;(\mathrm{mV}) $&145&207&301&408&575\\
			\hline
			\textbf{函数发生器}$ f\;(\mathrm{kHz}) $&\textbf{2.22}&\textbf{2.24}&\textbf{2.25}&\textbf{2.26}&\textbf{2.275}\\
			\hline
			\textbf{电压电流相位差}$ \varphi $&$ -0.116\pi $&$ -0.051\pi $&$ -0.035\pi $&$ 0.006\pi $&$ 0.067\pi $\\
			\hline
			\textbf{电阻电压}$ U_R\;(\mathrm{mV}) $&567&579&582&578&553\\
			\hline
			\textbf{函数发生器}$ f\;(\mathrm{kHz}) $&\textbf{2.30}&\textbf{2.38}&\textbf{2.43}&\textbf{2.62}&\textbf{3.18}\\
			\hline
			\textbf{电压电流相位差}$ \varphi $&$ 0.127\pi $&$ 0.240\pi $&$ 0.302\pi $&$ 0.368\pi $&$ 0.445\pi $\\
			\hline
			\textbf{电阻电压}$ U_R\;(\mathrm{mV}) $&530&413&301&154&62\\
			\hline
		\end{tabularx}
	\end{table}

	根据数据记录可作出$ \varphi-f,\;i-f $图象如下页。
	
	\begin{figure}[!h]
		\centering
		\includegraphics*[scale=0.35]{C_varphi-f.jpg}
		\caption{$ RLC $串联谐振电路相频曲线}
		\includegraphics*[scale=0.35]{C_i-f.jpg}
		\caption{$ RLC $串联谐振电路幅频曲线}
	\end{figure}

	由$ i-f $图象可读得谐振频率为$ f_0=2.241\,\mathrm{kHz} $,通频带宽为$ \Delta f=0.232\,\mathrm{kHz} $,故而可算得品质因子
	\[Q'=\frac{f_0}{\Delta f}=9.659\,\mathrm{kHz}\]
	
	可见$ Q_1,\;Q_2,\;Q' $与$ Q $差距均较大,可能的误差原因除读数时的数据跳变外,还有电感、电容的内阻使得实际实验电路与分析时的理想电路并不相同。
	
	\subsection{测$ RLC $并联电路的相频特性和幅频特性曲线}
	
	(1)调节函数发生器输出频率至并联部分电压$ u $与总电流相位相同,即达到谐振,此时可得谐振频率为
	\[f_p=2.249\,\mathrm{kHz}\]
	
	(2)在参考频率下测得实验数据如下\footnote{原始数据中$ \Delta t $部分数据符号与此表相反,原因在于实验时读得的是电流(CH2)与并联部分的电压$ u $(CH1$ - $CH2)的相位差,此处应进行取反。示波器光标功能仅能读取相同相位点间的时间间隔,根据$ \Delta t $计算相位差$ \varphi $的公式为
		\[\varphi=2\pi f\Delta t.\]}:
	\begin{table}[!h]		
		\centering
		\renewcommand\arraystretch{1.5}
		\caption{$ RLC $并联谐振电路实验数据记录}
		\begin{tabularx}{\textwidth}{|c|Y|Y|Y|Y|Y|Y|Y|}
			\hline
			$ f\;(\mathrm{kHz}) $&\textbf{2.05}&\textbf{2.15}&\textbf{2.20}&\textbf{2.231}&\textbf{2.24}&\textbf{2.247}&\textbf{2.25}\\
			\hline
			$ \Delta t\;(\mu\mathrm s) $&120&104&80&60&36&12&$ -2 $\\
			\hline
			$ \varphi $&$ 0.492\pi $&$ 0.447\pi $&$ 0.352\pi $&$ 0.268\pi $&$ 0.161\pi $&$ 0.053\pi $&$ -0.009\pi $\\
			\hline
			$ u\;(\mathrm{mV}) $&680&740&790&840&880&920&935\\
			\hline
			$ U_{R'}\;(\mathrm{mV}) $&684&401&193&103.8&94.4&81.92&79.36\\
			\hline
			$ f\;(\mathrm{kHz}) $&\textbf{2.253}&\textbf{2.2565}&\textbf{2.265}&\textbf{2.275}&\textbf{2.32}&\textbf{2.40}&\textbf{2.60}\\
			\hline
			$ \Delta t\;(\mu\mathrm s) $&$ -12 $&$ -22 $&$ -48 $&$ -60 $&$ -82 $&$ -94 $&$ -96 $\\
			\hline
			$ \varphi $&$ -0.054\pi $&$ -0.099\pi $&$ -0.217\pi $&$ -0.273\pi $&$ -0.380\pi $&$ -0.451\pi $&$ -0.499\pi $\\
			\hline
			$ u\;(\mathrm{mV}) $&935&925&895&845&755&680&550\\
			\hline
			$ U_{R'}\;(\mathrm{mV}) $&81.92&84.48&92.16&112.6&238&435&710\\
			\hline			
		\end{tabularx}
	\end{table}
	
	根据上表数据可作出如下图象:
	\begin{figure}[!h]
		\centering
		\includegraphics*[scale=0.35]{B_varphi-f.jpg}
		\caption{$ RLC $并联电路相频曲线}
	\end{figure}

	\newpage
	\begin{figure}[!h]
		\centering
		\includegraphics*[scale=0.35]{B_u-f.jpg}
		\caption{$ RLC $并联电路$ u-f $曲线}
		\includegraphics*[scale=0.35]{B_i-f.jpg}
		\caption{$ RLC $并联电路$ i-f $曲线}
	\end{figure}
	\subsection{观测$ RLC $串联电路的暂态过程}
	(1)调节$ R=0\,\Omega $,得到如下$ u_C $波形:
	\begin{figure}[!h]
		\centering
		\includegraphics*[scale=0.5]{R=0.png}
		\caption{阻尼振动状态波形}
	\end{figure}
	
	
	(2)自小到大调节$ R $的大小,当$ R=1300\,\Omega $时,波形振动部分消失,即可近似看作临界阻尼状态,波形如下:
	\begin{figure}[!h]
		\centering
		\includegraphics*[scale=0.5]{R=R_C.png}
		\caption{临界阻尼状态波形}
	\end{figure}
	
	根据实验所用电路元件参数可计算得到临界电阻理论值$ R_C $为
	\[R_C=\sqrt{\frac{4L}{C}}\approx1414\,\Omega\]
	实际测得的结果较小,除因临界阻值附近波形不明显影响选取的精度外,还有电路中电感、电容内阻带来的影响。
	
	(3)调节$ R $至过阻尼状态(2\,k$\Omega$,20\,k$\Omega$),在相应频率下记录$ u_C $波形:
	\begin{figure}[!h]
		\centering
		\includegraphics*[scale=0.25]{R=2k.png}
		\includegraphics*[scale=0.25]{R=20k.png}
		\caption{过阻尼状态下波形}
	\end{figure}
	\section{实验总结}
	本次实验的实验过程较为顺利,但对示波器的了解不足、实验原理的理解不够深入导致实验过程中还是屡有失误,譬如5.2节原始数据中$ \Delta t $的符号问题等。
\end{document}